%% USE one of these:
%% * the first when using pdflatex, which directly typesets your document in the
%%   chosen pdf/a format and you want to publish your thesis online,

%% * the second when you want to print your thesis to bind it, or
%% * the third when producing a ps file and a pdf/a from it.
%%
\documentclass[english, 12pt, a4paper, elec, utf8, a-1b, online]{aaltothesis}
%\documentclass[english, 12pt, a4paper, elec, utf8, a-1b]{aaltothesis}
%\documentclass[english, 12pt, a4paper, elec, dvips, online]{aaltothesis}

\UseRawInputEncoding
\usepackage{graphicx}
%% Math fonts, symbols, and formatting
\usepackage{amsfonts,amssymb,amsbsy,amsmath, enumitem}
 

%% THESIS INFO
\degreeprogram{Electrical engineering}

\major{Bioinformation technology}

\code{ELEC3016}

\univdegree{BSc}

\thesisauthor{Axel Hedman}

\thesistitle{Morning Fasting Glucose Assessed by Continuous Glucose Monitoring}

\place{Espoo}

\date{31.1.2023}

\supervisor{PhD\ Markus Turunen}

\advisor{D.Sc\ Lauri Palva}
\advisor{Mgr.\ Matej Kr\`{a}lik}


%% Aaltologo: syntax:
%% \uselogo{aaltoRed|aaltoBlue|aaltoYellow|aaltoGray|aaltoGrayScale}{?|!|''}
\uselogo{aaltoRed}{!}


%% THE ENGLISH ABSTRACT:
%% Thesis keywords:
\keywords{morning fasting glucose\spc metabolic health\spc CGM\spc fasting plasma glucose}

%% The abstract text. This text is included in the metadata of the pdf file as well
%% as the abstract page.
\thesisabstract{
TO DO!
}

%% Copyright text. Copyright of a work is with the creator/author of the work
%% regardless of whether the copyright mark is explicitly in the work or not.
%% You may, if you wish, publish your work under a Creative Commons license (see
%% creaticecommons.org), in which case the license text must be visible in the
%% work. Write here the copyright text you want. It is written into the metadata
%% of the pdf file as well.
%% Syntax:
%% \copyrigthtext{metadata text}{text visible on the page}

\copyrighttext{Copyright \noexpand\copyright\ \number\year\ \ThesisAuthor}
{Copyright \copyright{} \number\year{} \ThesisAuthor}


\begin{document}

%% Create the coverpage
\makecoverpage


%% Typeset the copyright text.
%% If you wish, you may leave out the copyright text from the human-readable
%% page of the pdf file. This may seem like a attractive idea for the printed
%% document especially if "Copyright (c) yyyy Eddie Engineer" is the only text
%% on the page. However, the recommendation is to print this copyright text.
\makecopyrightpage


%% ENGLISH ABSTRACT
%% All the details (name, title, etc.) on the abstract page appear as specified
%% above.
\begin{abstractpage}[english]
    \abstracttext{}
\end{abstractpage}

%% Force new page so that the Swedish abstract starts from a new page
\newpage

%% SWEDISH ABSTRACT.
\thesistitle{Morgonfasteglukos fastst\"{a}llt av kontinuerlig glukosmonitorering}
\supervisor{TkD\ Markus Turunen}
\advisor{?\ Lauri Palva} 
\advisor{Mgr. Matej Kr\`{a}lik} 
\degreeprogram{Elektroteknik}
\major{Bioinformationsteknologi}
\keywords{morgonfasteglukos\spc metabol h\"{a}lsa\spc CGM\spc fasteplasmaglukos}
\begin{abstractpage}[swedish]
TO DO!
\end{abstractpage}


%% PREFACE
%% This section is optional.
%%\vspace{5mm}
%%{\hfill Axel J.\ H.\ Hedman \hspace{1cm}}

%%\newpage


%% TABLE OF CONTENTS
\thesistableofcontents


%% SYMBOLS AND ABBREVIATIONS
\mysection{Symbols and abbreviations}

\subsection*{Symbols}

\begin{tabular}{ll}

\end{tabular}

\subsection*{Abbreviations}

\begin{tabular}{ll}
BMI         & body mass index \\
CGM      & continuous glucose monitoring \\
FPG        & fasting plasma glucose \\
MARD        &  mean absolute relative difference \\
MFG         & morning fasting glucose \\
OGTT        & Oral Glucose Tolerance Test \\
\end{tabular}

\cleardoublepage

%% List of research questions
\newlist{questions}{enumerate}{2}
\setlist[questions,1]{label=RQ\arabic*.,ref=RQ\arabic*}
\setlist[questions,2]{label=(\alph*),ref=\thequestionsi(\alph*)}

%% INTRODUCTION
\section{Introduction}
\thispagestyle{empty}

%% TEXT %%
The prevalence of obesity and overweight is a growing 
health issue worldwide. Since 1980, the rate of obesity 
and overweight has doubled. Today, nearly a third of 
the world's population is considered obese or overweight 
\cite{chooi_epidemiology_2019}. As the prevalence of obesity and overweight 
increases, public health is at risk. Obesity affects 
nearly all physiological activity in the human body. It 
lowers life expectancy and increases the risk of multiple 
severe diseases including diabetes mellitus, 
cardiovascular disease, and various types of cancer\cite{chooi_epidemiology_2019}\cite{nammi_obesity_2004}. 
There are several approaches how to tackle obesity and 
overweight but the majority of them revolve around 
diet and lifestyle management. Medications and surgery may solve 
short-term complications but several studies have shown 
that lifestyle changes are necessary to ultimately 
overcome obesity\cite{nammi_obesity_2004}\cite{mauro_barriers_2008}\cite{powell_effective_2007}\cite{marquis-gravel_intensive_2015}.

When used correctly, glucose monitoring can be a useful tool when tracking 
changes in lifestyle. Diet and exercise heavily affect blood
glucose levels as well as overall health. Studies have shown that healthy lifestyle changes
not only lowers blood glucose levels but it also results in a healthier Body Mass Index (BMI) and weight,
improved cholesterol and triglyceride values, and lower blood pressure\cite{yamaoka_effects_2012}.
If not managed correctly, blood glucose levels can reach unusually high or low levels potentially causing 
acute and chronic conditions\cite{mathew_blood_2022}.  

Traditionally, glucose levels have been monitored with blood 
samples and fingertip pricks. A combination of obesity
and unnecessarily high glucose levels can lead to the onset of type 2 diabetes\cite{garber_obesity_2012}.
Type 2 diabetes patients are heavily dependent on glucose monitoring 
to manage possible medications and avoid critical conditions\cite{ripsin_management_2009}. 
Nevertheless, studies have shown that
with appropriate lifestyle changes, type 2 diabetes can be reversed with diet alone\cite{taylor_type_2013}. 
Such a reversal requires an understanding of glucose management.
Hence, glucose management is a vital part of alleviating and 
preventing obesity and type 2 diabetes.

Fasting Plasma Glucose (FPG) is an established indicator of whether 
an individual is metabolically healthy or not\cite{moebus_impact_2011}. 
FPG is usually measured from a blood sample taken after a minimum of 8 
hours of fasting\cite{the_expert_committee_on_the_diagnosis_and_classification_of_diabetes_mellitus_report_1997}. 
An FPG test traditionally requires a medical appointment. But what if there was a
way of estimating FPG values without having to schedule a medical 
appointment? 

Modern technology has brought alternative glucose measurement methods. 
Continuous Glucose Monitoring (CGM) is an already proven method to 
monitor glucose levels\cite{danne_international_2017}. CGM 
has struggled to become a medically approved method due to
large measurement errors. In the year 2000, the Mean Absolute Relative Difference
(MARD) of CGM was larger than $\pm$20\% which is the accepted average for 
regulatory approval. As technology advanced the MARD of CGM has 
now reduced to under $\pm$10\% and accuracy continues to improve\cite{rodbard_continuous_2016}\cite{ctx3255462370006526}.

CGM uses an electrochemical biosensor to measure glucose levels
in the interstitial fluid between the skin and blood capillaries.
The biosensor utilizes a small filament inserted through the skin 
to measure glucose levels. The in vivo response to a needle-type 
biosensor has been one of the biggest challenges to accurate CGM
measurements. The infiltration of proteins, the release of cytokines 
and reactive oxygen species, and subsequent influx of 
inflammatory cells all contribute to the instability of implanted 
glucose sensors.\cite{ctx3255462370006526}. Nevertheless, compared to
blood samples and fingertip pricks, CGM provides a complete set 
of continuous glucose data for the user to analyze. A fingertip
prick, for example, doesn't provide an overview of the long-term
effects that lifestyle and diet have on glucose levels. Hence, CGM
has established itself as a viable option when it comes to 
glycemic control and overall improvement of metabolic health. 

This thesis is the result of a collaboration 
with Veri\cite{noauthor_veri_nodate}, a company focusing on improving 
metabolic health through CGM. Veri combines a mobile application with a CGM 
sensor providing continuous glucose data for the user to analyze. By tracking 
glucose levels with a CGM, users are able to observe the metabolic impact
of diet, sleep, stress, exercise, and other lifestyle factors on long-term health. 
To support the research and conclusions of this thesis, Veri has 
provided user data of 6000 people with an average of 7 days of data 
per user. The dataset includes meals, exercise, and sensor data. 
In addition to the dataset, Veri provides guidance and insights to match
industry standards in real-world applications.

The aim of this thesis is to explore the possible usage of CGM 
data for approximating FPG. After establishing whether FPG can be 
estimated, the thesis will also look into how fasting glucose values 
vary over time and if they can be used to estimate the prevalence of 
certain metabolic disorders. In science, the term is Fasting Plasma Glucose but with CGM and without 
complete user data it is to achieve a value as accurate as 
FPG. Therefore, Veri has established morning fasting glucose (MFG) to estimate FPG. 
MFG aims to be as close to FPG as possible. The advantage
of MFG is that, with CGM, it can be continuously acquired from day to day without 
having to schedule a doctor's appointment. Likewise, the estimation includes data 
acquired over a longer period of time compared to the momentary measurement of FPG. 
Earlier research on the topic and the dataset provided by Veri will 
be the supporting pillars when examining the possibilities of estimating an 
FPG value from CGM data. 

The research questions addressed in this thesis are
\begin{questions}[leftmargin=50pt]
    \item Can morning fasting glucose be estimated using CGM data?
    \item How does morning fasting glucose of a single individual vary over time?
    \item Can MFG be used to estimate the prevalence of certain metabolic disorders? \label{itm:qwithlabel}
\end{questions}

In chapters 2,3 and 4...

%% END TEXT %%

\clearpage


%% BACKGROUND
\section{Background}
%% TEXT %%
\subsection{Fasting Plasma Glucose}
FPG is a metric used when evaluating the metabolic health of an individual.
FPG is often used as a screening method to evaluate the risk of developing type 2 diabetes.
Normal FPG levels range between 4.0 mmol/l and 5.6 mmol/l. An FPG value of 5.6 - 6.0 mmol/l 
indicates a risk of developing type 2 diabetes. Prediabetes is a state in which FPG
levels exceed normal levels to reach values of 6.1 - 6.9 mmol/l. FPG values
higher than 7 mmol/l indicate that the individual has diabetes\cite{mathew_blood_2022}. 
If FPG levels reach the prediabetes state it is recommended to conduct further testing to confirm if 
the patient has diabetes or not. If not the patient should continue frequent screening and apply 
lifestyle changes to ensure that FPG ceases to increase.\cite{ekoe_screening_2018}.
Type 2 diabetes develops over a longer period of time. Impaired Glucose Tolerance (IGT) and Impaired Fasting
Glucose (IFG) are both complications associated with diabetes\cite{walker_diet_2010}. 

\subsubsection{Impaired Fasting Glucose \& Impaired Glucose Tolerance}
IGF is defined by FPG concentration. An elevated FPG value of 5.6-7 mmol/l is enough to diagnose 
a patient with IFG \cite{nathan_impaired_2007}. IGT is determined by an Oral Glucose Tolerance 
Test (OGTT). OGTT includes consuming a 75g glucose load and measuring plasma glucose 2h after consumption. 
An OGTT value of 7.8-11.1 mmol/l and an FPG value of < 7 mmol/l is enough to diagnose a patient with IGT\cite{nathan_impaired_2007}. 
The overlap in FPG shows that one can suffer from both IFG and IGT. Historical patterns show that among individuals 
diagnosed with either IGT or IFG, 25\% progress to diabetes, 50\% remain in the abnormal range 
without developing diabetes, and 25\% reacquire normal glucose tolerance. 
Being diagnosed with both of these metabolic complications doubles the rate of developing diabetes \cite{nathan_impaired_2007}.


\subsection{Lifestyle and plasma glucose}
The earlier one detects FPG abnormalities the better. Metabolic disorders develop over a longer period of time.

\subsubsection{Diet}

\subsubsection{Excersice}

\subsubsection{Stress}

\subsubsection{Obesity and type 2 diabetes}
Obesity is a result of an unhealthy lifestyle over a longer period of time. If not treated correctly, obesity may cause elevations in FPG which leads to IGT and IFG, and eventually diabetes[SOURCE]. 
Studies have shown that correctly implemented lifestyle changes during the prediabetes stage can lower FPG levels and reduce the risk for type 2 diabetes significantly\cite{walker_diet_2010}\cite{taylor_type_2013}. 
Other studies have also shown how type 2 diabetes can be reversed by diet\cite{taylor_type_2013}. A strict low calory diet
has been proven to bring FPG to a normal level among individuals diagnosed with type 2 diabetes\cite{taylor_type_2013}. 

\subsection{Continuous Glucose Monitoring}

%% END TEXT %%
\clearpage


%% RESEARCH MATERIAL AND METHODS
\section{Research material and methods}
%% TEXT %%

%% END TEXT %%
\clearpage


%% RESULTS
\section{Results}
%% TEXT %%

%% END TEXT %%
\clearpage


%% SUMMARY
\section{Summary} 
%% TEXT %%

%% END TEXT %%
\clearpage


%% REFERENCES
\bibliographystyle{IEEEtran}
\bibliography{ref}

\clearpage

%% ADD POSSIBLE APPENDIX HERE
\thesisappendix

\end{document}
